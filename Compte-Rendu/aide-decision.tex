\documentclass[12pt]{article}

\usepackage{ucs}
\usepackage[utf8x]{inputenc}
\usepackage[frenchb]{babel}
%\usepackage{fontenc}

\usepackage{fancyhdr, graphicx}
\usepackage[normalem]{ulem}
\usepackage{fullpage}

\newcommand*\tick{\item[\Checkmark]}
\newcommand*\fail{\item[\XSolidBrush]}
\newcommand*\bulet{\item[$\bullet$]}
\newcommand*\diamont{\item[$\diamond$]}

\title{Projet d'aide à la décision \\ Compte rendu client.}
\date{\today}
\author{Ségolène Minjard \and Salma El-Alaoui-Talibi \and Quentin Dupont \and Donovan Fournier \and Benjamin Legrand \and Zied Thabet \and Hexanôme H4304 \\ \\ Enseignants: Maryvonne \textsc{Miquel}
et Anne \textsc{Legait}  }

\begin{document}

\maketitle

\vspace{25pt}
\setcounter{tocdepth}{2}
\tableofcontents % needs 2compilations
\pagebreak

\section{Introduction}
Ce document est un compte rendu de l'étude menée par notre hexanôme H4304 pour le compte de la société FaBrique. Cette étude a pour but d'optimiser l'utilisation des ressources de l'entreprise en proposant le meilleur plan de production à mettre en place. \newline
En premier lieu, nous considèrerons indépendamment les objectifs métiers de chacun des cadres impliqués dans cette étude d'aide à la décision pour apporter une solution optimale à chacun.
En second lieu, nous proposons au responsable d'entreprise une solution qui satisfait au mieux l'ensemble des points de vue.
Finalement, nous dégagerons parmi les 8 propositions de gestion de l'atelier, la solution optimale au vu des critères définis par les décideurs de l'entreprise.    


\section{Prise en compte des contraintes globales}
Deux critères nous semblent limitants d'un point de vue de la production. En effet, les machines ne peuvent fonctionner qu'en présence des ouvriers et leurs horaires de travail sont limités. Il convient donc de limiter les temps d'utilisation de chacune d'elles.
En prenant en compte les durées de production respectives de chacun des produits sur les différentes machines, on introduit les 7 contraintes suivantes :
\begin{center}
$\mathbf{ 
   \lbrace
   \begin{array}{c}
      8 * x_{A} + 15 * x_{B} + 0 * x_{C} + 5 * x_{D} + 0 * x_{E} + 10 * x_{F}\leq 4800 \\
      7 * x_{A} + 1 * x_{B} + 2 * x_{C} + 15 * x_{D} + 7 * x_{E} + 12 * x_{F} \leq 4800 \\
      8 * x_{A} + 1 * x_{B} + 11 * x_{C} + 0 * x_{D} + 10 * x_{E} + 25 * x_{F} \leq 4800 \\
      2 * x_{A} + 10 * x_{B} + 5 * x_{C} + 4 * x_{D} + 13 * x_{E} + 7 * x_{F} \leq 4800 \\
      5 * x_{A} + 0 * x_{B} + 0 * x_{C} + 7 * x_{D} + 10 * x_{E} + 25 * x_{F} \leq 4800 \\
      5 * x_{A} + 5 * x_{B} + 3 * x_{C} + 12 * x_{D} + 8 * x_{E} + 0 * x_{F} \leq 4800 \\
      5 * x_{A} + 3 * x_{B} + 5 * x_{C} + 8 * x_{D} + 0 * x_{E} + 7 * x_{F} \leq 4800 \\
   \end{array}
 } $ 
\end{center}

De plus, la production est limitée par les matières premières. De fait, en appliquant les pondérations associées, on trouve :
\begin{center}
$\mathbf{
   \lbrace
   \begin{array}{c}
      1 * x_{A} + 2 * x_{B} + 1 * x_{C} + 5 * x_{D} + 0 * x_{E} + 2 * x_{F}\leq 350 \\
      2 * x_{A} + 2 * x_{B} + 1 * x_{C} + 2 * x_{D} + 2 * x_{E} + 2 * x_{F} \leq 620 \\
      1 * x_{A} + 0 * x_{B} + 3 * x_{C} + 2 * x_{D} + 2 * x_{E} + 0 * x_{F} \leq 485 \\
   \end{array}
 } $ 
\end{center}

Par ailleurs, nous devons ajouter des contraintes de domaines. En effet, la production ne peut pas être négative. Nous avons donc :
$ x_{A} \geq 0$, $x_{B}  \geq 0$, $x_{C} \geq 0$, $x_{D} \geq 0$, $x_{E} \geq 0$, $x_{F} \geq 0 $
Des contraintes relatives aux différents cas d'analyses seront ajoutées aux précédentes par la suite.

\section{Analyse des objectifs de chacun des cadres}
Dans cette partie, on met en oeuvre des techniques de programmation linéaire monocritère afin de proposer individuellement à chaque cadre la solution qui permet d'optimiser son objectif métier.
\subsection{Le comptable}
L'objectif du comptable est de maximiser le bénéfice de l'entreprise. Ce bénéfice est calculé en tenant compte des coûts de fonctionnement des machines et du coût d'achat des matières premières. Pour chaque matière le bénéfice unitaire (i.e. pour un produit vendu) sera donc donné par la formule suivante : \begin{center} $pv - \sum_{i=1}^{3}(paMP_{i} * qteMP_{i}) - \frac{1}{60} \sum_{j=1}^{7}(tupM_{j} * chM_{j}) $ \end{center} avec: \newline 
\begin{description}
\item[pv: ]Prix de vente du produit considéré fini.
\item[$\mathbf{paMP_{i} :}$] Le prix d'achat de la matière première de numéro i.
\item[$\mathbf{qteMP_{i} :}$] La quantité de matière première i, nécessaire pour la fabrication du produit considéré.
\item[$\mathbf{tupM_{j} :}$] Temps unitaire d'usinage du produit considéré sur la machine j.
\item[$\mathbf{chM_{j} :}$] Coût horaire de la machine j.
\end{description}
Il en ressort le problème de programmation linéaire monocritère suivant : \newline 
\noindent\fbox{\parbox{\linewidth\fboxrule-2\fboxsep}{
Maximiser $ Z_{comptable}= 5.67*x_{A} +11.88*x_{B} +12.27*x_{C} +1.03*x_{D} +31.65*x_{E} +27.55*x_{F}$ 
\\ 
sous les contraintes : \\
:D Contraintes :D  }}
\\
On peut à partir des coefficients de la fonction objectif classer les produits selon le bénéfice unitaire qu'ils permettent de dégager. On obtient dans l'ordre décroissant de bénéfice unitaire : E, F, C, B, A, D.
\\
La résolution mathématique de ce problème de programmation linéaire donne les résultats suivant :\\
\begin{center}
$\mathbf{X^{*}_{comptable} = 
   \left (
   \begin{array}{c}
      x_{A} = 0 \\
      x_{B} = 20.41 \\
      x_{C} = 0 \\
      x_{D} = 0 \\
      x_{E} = 242.50 \\
      x_{F} = 94.18 \\
   \end{array}
   \right )
 } $ 
\end{center}
On constate que les produits les plus rentables (E puis F) sont beaucoup produits. Les produits
les moins rentables (A et D) ne sont pas du tout produits. Le produit C est légèrement plus rentable que le produit B mais B est préféré à C car il n'utilise pas la matière première 3 qui est beaucoup utilisée par E (présence d'une contrainte sur l'utilisation des matières premières).
\\
\begin{center}
\textbf{Bénéfice maximal = 10512}
\end{center}

Satisfaction des contraintes : 
\begin{center}
$\mathbf{A.X^{*}_{comptable} - b = 
   \left (
   \begin{array}{c}
      -3552 \\
      -1952 \\
      0 \\
      -784 \\
      -20 \\
      -2758 \\
      -4080 \\
      -121 \\
      0 \\
      0\\
   \end{array}
   \right )
 } $ 
\end{center}
Les 7 premières lignes correspondent à l’utilisation des machines. On constate que les
machines 1 et 7 sont les moins utilisées. Ceci s’explique par le fait que ces machines ne sont
pas nécessaires pour produire E qui est le produit le plus réalisé. La machine 3 est utilisée au maximum car elle est très utilisée par les produits les plus réalisés (E et F). Les matières premières 2 et 3 sont consommées au maximum. En effet ce sont les matières premières nécessaires à la production de E. La production de E est ainsi poussée au maximum (jusqu'à épuisement des matières premières nécessaires).

\subsection{Le responsable d'atelier}
T
\subsection{Le responsable des stocks}
O
\subsection{Le responsable commercial}
L'objectif du commercial est de conserver une certaine diversité dans sa gamme de produits tout en produisant le plus de produits. Concrètement, l’entreprise doit produire autant de produit de la gamme 1 (composée des produits A,B,C) que de la gamme 2 (composée des produits D,E,F).
On peut donc en déduire une contrainte d’égalité :\begin{center} $(x_{a} + x_{b} + x_{c}) - (x_{d} + x_{e} + x_{f}) = 0$ \end{center}

De plus, il nous semble légitime de rajouter dans notre modèle le souhait de produire le plus de produits (comme dans le modèle du responsable atelier).
On cherchera donc à maximiser f tel que :\begin{center} $f_{commercial}(X)=x_{a} + x_{b} + x_{c} + x_{d} + x_{e} + x_{f}$ \end{center}
La résolution mathématique de ce problème de programmation linéaire donne les résultats suivant :\\
\begin{center}
$\mathbf{X^{*}_{commercial} = 
   \left (
   \begin{array}{c}
      x_{A} = 142.12 \\
      x_{B} = 0 \\
      x_{C} = 44.42 \\
      x_{D} = 0 \\
      x_{E} = 104.81 \\
      x_{F} = 81.73 \\
   \end{array}
   \right )
 } $ 
\end{center}
On vérifie facilement que la contrainte inhérente au problème du commercial est respecté : le lot de produits A et C est produits en même quantité que le lot de produits E et F.
On remarque que les produits B et D sont totalement supprimés de la production. Cette particularité montre que le modèle B demande plus de ressources et/ou de temps de travail que ses modèles équivalents (A et C) et que d'un point de vue quantitatif, il est préférable de choisir ces derniers. 

\begin{center}
\textbf{Nombre de produits total = 373.08}
\end{center}

Satisfaction des contraintes : 
\begin{center}
$\mathbf{A.X^{*}_{commercial} - b = 
   \left (
   \begin{array}{c}
      -2846 \\
      -2002 \\
      -83 \\
      -2359 \\
      -998 \\
      -3118 \\
      -32952 \\
      0 \\
      0 \\
      0\\
   \end{array}
   \right )
 } $ 
\end{center}
Ces valeurs nous indiquent que les facteurs limitants de notre problème sont les ressources de productions. Une augmentation de ces dernières augmenterai le nombre maximum de produits.
Dans ces conditions, la machine 7 est sous-exploitée tandis que la 3ème est presque en constante utilisation. Ces constatations sont explicables par l'importante productions des modèles A, E et F. 

\subsection{Le responsable du personnel}
O
%\begin{figure}
%	\begin{center}
%	\includegraphics[scale=0.7]{UML.png}
%	\end{center}
%	\caption{Diagram label}
%\end{figure}

%\pagebreak


%\include{partie_1} % \include == \clearpage + \input


\end{document}
